\documentclass[a4,12pt]{article}
\usepackage[utf8]{inputenc}
\usepackage[spanish]{babel}
\usepackage[margin=3cm]{geometry}
\usepackage{graphicx}
\usepackage{import}
\usepackage{color}
\usepackage{verbatim}
\usepackage{listings}
\usepackage[normalem]{ulem}

\usepackage[hidelinks]{hyperref}

\title{Comparación de Algoritmos de Ordenación}
\author{Vicente Javier Vidal-Abarca González}
\date{12 de junio de 2017}

\begin{document}

\maketitle

\begin{abstract}

Este documento, ha sido creado en \LaTeX, con el fin mostrar el conocimiento que poseo para crear un documento con dicha herramienta. Además, de guardar un registro de control usando GIT. Para demostrarlo, voy a realizar un pequeño codigo en python, que comparará el tiempo que tarda en ordenar una lista de un tamaño predefinido, usando tanto el algoritmo de Burbuja, como el Quicksort.

\end{abstract}

\newpage
\tableofcontents
\newpage

\section{Introducción}
Bienvenido a este documento. Como bien esta explicado en el resumen, este documento ha sido realizado en \LaTeX, y con el mostraré el desarrollo de esta práctica.

\subsection{Programas usados}
En esta práctica, he usado 3 programas principalmente:
\begin{itemize}

\item GIT: Repositorio usado para almacenar las pruebas de control de la práctica.

\item \LaTeX: Herramienta usada para crear este documento PDF eficientemente.

\item Python IDLE: Entorno donde he desarrollado el código y las pruebas necesarias para realizar este proyecto.

\end{itemize}



\end{document}
